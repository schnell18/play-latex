\documentclass[a4paper]{article}
%\documentclass{article}
% if you don't load a package like geometry or hyperref (who doesn't use hyperref?), then you could add these two lines
\pdfpagewidth=\paperwidth
\pdfpageheight=\paperheight


\makeatletter
\newcommand\usemm[1]{%
  \strip@pt\dimexpr0.3514598\dimexpr #1\relax\relax mm%
}
\newcommand\usein[1]{%
  \strip@pt\dimexpr0.013837\dimexpr #1\relax\relax in%
}
\makeatother

\begin{document}
\the\paperheight \par
\the\paperwidth  \par
\the\hsize       \par % or \textwidth
\the\vsize       \par % or \textheight

\usemm{\the\paperheight} \par
\usemm{\the\paperwidth}  \par
\usemm{\the\hsize}       \par
\usemm{\the\vsize}       \par

\usein{\the\paperheight} \par
\usein{\the\paperwidth}  \par
\usein{\the\hsize}       \par
\usein{\the\vsize}       \par
\end{document}