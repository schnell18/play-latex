\documentclass{article}
\usepackage{forest}
\usetikzlibrary{arrows.meta,shapes,positioning,shadows,trees}
%
\tikzset{
    basic/.style  = {draw, text width=2cm, drop shadow, font=\sffamily, rectangle},
    root/.style   = {basic, rounded corners=2pt, thin, align=center,
                     fill=green!30},
    onode/.style = {basic, thin, rounded corners=2pt, align=center, fill=green!60,text width=3cm,},
    tnode/.style = {basic, thin, align=left, fill=pink!60, text width=6.5em},
    edge from parent/.style={draw=black, edge from parent fork right}
}
%
\begin{document}
\begin{forest} for tree={
    grow=east,
    growth parent anchor=east,
    parent anchor=east,
    child anchor=west,
    edge path={\noexpand\path[\forestoption{edge},->, >={latex}] 
         (!u.parent anchor) -- +(5pt,0pt) |- (.child anchor)
         \forestoption{edge label};}
}
[Drawing diagrams, root
    [Defining node and arrow styles, onode
        [Setting shape, tnode]
        [Choosing color, tnode]
        [Adding shading, tnode] ]
    [Positioning the nodes, onode
        [Using a Matrix, tnode]
        [Relatively, tnode]
        [Absolutely, tnode] 
        [Using overlays, tnode] ]
    [Drawing arrows between nodes, onode
        [Default arrows, tnode]
        [Arrow library, tnode]
        [Resizing tips, tnode] 
        [Shortening, tnode]
        [Bending, tnode] ] ]
\end{forest}
\end{document}