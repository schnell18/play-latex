\documentclass{article}
\usepackage[utf8]{inputenc}
\usepackage{tcolorbox}
\usepackage{minted}
\tcbuselibrary{minted,skins}

% Define custom styles for different code themes
\newtcblisting{gocode}[1][]{
    listing engine=minted,
    listing only,
    minted language=go,
    minted options={
        linenos=true,
        numbersep=8pt,
        xleftmargin=0pt,
        fontsize=\footnotesize,
        breaklines=true,
        breakanywhere=true,
        highlightlines={3,8},
        highlightcolor=yellow!20
    },
    colback=white,
    colframe=blue!70!black,
    boxrule=1pt,
    arc=3pt,
    outer arc=3pt,
    left=15pt,
    right=5pt,
    top=5pt,
    bottom=5pt,
    drop shadow,
    #1
}

\newtcblisting{gocode-bad}[1][]{
    listing engine=minted,
    listing only,
    minted language=go,
    minted style=tango,
    minted options={
        linenos=true,
        numbersep=8pt,
        xleftmargin=0pt,
        fontsize=\footnotesize,
        breaklines=true,
        frame=leftline,
        framesep=2mm,
        rulecolor=\color{red!70!black}
    },
    enhanced,
    colback=red!5,
    colframe=red!70!black,
    boxrule=1pt,
    arc=3pt,
    outer arc=3pt,
    left=18pt,
    right=5pt,
    top=5pt,
    bottom=5pt,
    #1
}

\newtcblisting{gocode-elegant}[1][]{
    listing engine=minted,
    listing only,
    minted language=go,
    minted style=tango,
    minted options={
        linenos=true,
        numbersep=8pt,
        xleftmargin=0pt,
        fontsize=\footnotesize,
        breaklines=true,
        frame=leftline,
        framesep=2mm,
        rulecolor=\color{green!50!black}
    },
    enhanced,
    colback=green!5,
    colframe=green!50!black,
    boxrule=1pt,
    arc=3pt,
    outer arc=3pt,
    left=18pt,
    right=5pt,
    top=5pt,
    bottom=5pt,
    #1
}

\begin{document}

\title{Go Code Examples with Diverse Highlighting}
\maketitle

\section{GitHub Style (Light Theme)}
Clean and professional appearance with line highlighting:

\begin{gocode}[title={\textbf{Web Server Example}}]
package main

import (
    "fmt"
    "log"
    "net/http"
    "time"
)

func handler(w http.ResponseWriter, r *http.Request) {
    fmt.Fprintf(w, "Hello, %s! Current time: %s",
                r.URL.Path[1:], time.Now().Format(time.RFC3339))
}

func main() {
    http.HandleFunc("/", handler)
    log.Fatal(http.ListenAndServe(":8080", nil))
}
\end{gocode}

\section{Elegant Minimal Style}
Subtle left border with tango color scheme:

\begin{gocode-elegant}[title={\textbf{Concurrent Processing Example}}]
package main

import (
    "fmt"
    "sync"
    "time"
)

func worker(id int, wg *sync.WaitGroup) {
    defer wg.Done()
    fmt.Printf("Worker %d starting\n", id)
    time.Sleep(time.Second)
    fmt.Printf("Worker %d done\n", id)
}

func main() {
    var wg sync.WaitGroup
    for i := 1; i <= 3; i++ {
        wg.Add(1)
        go worker(i, &wg)
    }
    wg.Wait()
    fmt.Println("All workers completed")
}
\end{gocode-elegant}

\section{Bad code Style}
Subtle left border with tango color scheme:

\begin{gocode-bad}[title={\textbf{Concurrent Processing Example}}]
// Package calculator provides mathematical operations for basic arithmetic.
// It supports addition, subtraction, multiplication, and division operations
// with proper error handling for edge cases like division by zero.
package calculator

import "fmt"

// Add returns the sum of two integers.
// It handles overflow by returning the mathematical result without bounds checking.
func Add(a, b int) int {
	return a + b
}

// Divide performs integer division and returns an error for division by zero.
func Divide(a, b int) (int, error) {
	if b == 0 {
		return 0, fmt.Errorf("cannot divide %d by zero", a)
	}
	return a / b, nil
}
\end{gocode-bad}
\end{document}
